\documentclass[t]{beamer}
\usepackage[T1]{fontenc}
\usepackage{libertinus}
\usepackage[scaled=0.85]{DejaVuSansMono}
\usepackage{hyperref}
\hypersetup{
colorlinks=true,
linkcolor=blue,
linkbordercolor=blue,
urlcolor=blue,
urlbordercolor=blue,
}
\usepackage{svg}
\usepackage{wrapfig}
\usepackage{minted}
\usepackage{verbatim}
\usepackage{soul}
\usepackage{upquote}

\usetheme{Boadilla}
\setbeamertemplate{navigation symbols}{}
\setbeamertemplate{enumerate items}[circle]

\makeatother
\setbeamertemplate{footline}{%
\hbox{%
\begin{beamercolorbox}[wd=1.0\paperwidth,ht=2.25ex,dp=1ex]{title in head/foot}%
\hspace*{1em}%
\usebeamerfont{title in head/foot}\NoHyper\insertshorttitle\endNoHyper%
\hfill%
\insertframenumber{} / \inserttotalframenumber
\hspace*{1em}%
\end{beamercolorbox}}%
}
\makeatletter

\setbeameroption{show notes on second screen=right}
%\setbeameroption{hide notes}
\setbeamertemplate{note page}{\insertnote\par}

\title{Rust and functional programming}
\author{Isaac Elliott}
\date{19 September, 2023}

\begin{document}


\frame{\titlepage}


\begin{frame}[fragile]
\frametitle{What is Rust?}

\begin{columns}

\begin{column}[T]{0.75\textwidth}

\url{https://www.rust-lang.org/}

\begin{itemize}
\item Created at Mozilla in late 2000s and early 2010s
\item Prioritises safety and performance
\item Built on modern (30 years or younger) programming language research
\end{itemize}

\end{column}

\begin{column}[T]{0.25\textwidth}
\includesvg{images/rust-logo-blk.svg}
\end{column}

\end{columns}

\vfill

\begin{block}{Hello world}
\begin{minted}{rust}
fn main() {
    println!("Hello world!");
}
\end{minted}
\end{block}

\end{frame}


\begin{frame}[c,fragile]
\frametitle{What is functional programming?}

\begin{itemize}
\item Programming with functions?
\item ...with higher-order functions?
\item 
\item Haskell? 
\end{itemize}

\note{I find functional programming a lot harder to define than something like Rust.}
\end{frame}


\begin{frame}[c,fragile]
\begin{center}
\usebeamerfont{title}{\usebeamercolor[fg]{frametitle}{Personal identity and functional programming}}
\end{center}
\note<+>{
Before I get into the technical details, I want to talk about a use (misuse?) of the
term "functional programming": as a sort of "tribal identity marker". Here's something
true of my experience that might also be true for you:

\medskip

When I choose between similar things, I have a tendency toward seeing my choice as
somehow, objectively, "more right" than the alternatives. Usually this tendency is
incorrect, and it takes a conscious effort for me to see that.

\medskip

I also enjoy spend time with people who have made similar choices. I like talking about
programming with other programmers (that's why I'm here tonight!), and listening to music
with people who have similar tastes. In situations like these, in a community of sorts,
I feel a kind of satisfaction that I don't feel when I'm alone or with people who are
extremely different to me.

\medskip

All this is to say that I find shared context / community / meeting people with similar
interests to be \emph{very} meaningful.
}

\note<+>{
I've also experienced something kind of toxic that happens when this (usually incorrect)
tendency to think that my choice is Better spreads to the group and becomes part of the
way we bond. We start to celebrate the supposed fact that we are the small group who
made the "right" choice, and start to find meaning in pointing out the ways that everyone
else is "wrong".

\medskip

This is kind of how I related to "functional programming" when I was younger. I enjoyed a
particular style of programming or flavour of programming language, and thought I was
being attracted to some kind of objective superiority. From that perspective, it was easy to
be a part of discussions about the "inferiority" of other styles of programming and other
flavours of programming language. Not critique, which is done in good faith with the intent
of improving the object of critique, but a sort of destructive and dismissing criticism, a
perverse pleasure in noting how others have failed to live up to the standard of "functional
programming".

\medskip

I failed to see the ways in which my attraction to "functional programming" was due to
my personal history and my taste in writing software. And I failed to acknowledge the
nebulousity of this "paradigm" which at the time seemed so objective.
}
\end{frame}


\begin{frame}[c,fragile]
\frametitle{Common functional programming concepts}

\begin{itemize}
\item Algebraic datatypes and higher-order functions
\item Immutability
\item Static types
\item Equational reasoning
\end{itemize}

\note{
So instead of presenting a definition of functional programming and measuring
Rust against it, I'm going cover a few topics that seem most closely associated to
"functional programming" (whatever \emph{that} is), and talk about the extent to which they show
up in Rust.

\medskip

Before I reveal them, I want to do a survey. What are some of the ideas \emph{you} would include?
}
\end{frame}


\begin{frame}[fragile]
\frametitle{Algebraic datatypes and higher-order functions --- ADTs}

\begin{columns}

\begin{column}[T]{0.48\textwidth}
\begin{block}{Definition}
\small
\begin{minted}{rust}
enum Sum<A, B>{
    Left(A),
    Right(B)
}

struct Product<A, B>{
    first: A,
    second: B
}
\end{minted}
\end{block}
\end{column}

\begin{column}[T]{0.48\textwidth}
\begin{block}{Creation}
\small
\begin{minted}{rust}
let x = Sum::Left(1);
let y = Sum::Right(true);

let z = Product{
    first: "x",
    second: "y"
};


\end{minted}
\end{block}
\end{column}

\end{columns}

\begin{block}{Use}
\begin{columns}
\begin{column}[T]{0.48\textwidth}
\small
\begin{minted}{rust}
let x: Sum<A, B> = ...;
let y = match x {
    Sum::Left(a) => ...,
    Sum::Right(b) => ...
};
\end{minted}
\end{column}
\begin{column}[T]{0.48\textwidth}
\small
\begin{minted}{rust}
let x: Product<A, B> = ...;
let y = x.first;
let z = y.second;
\end{minted}
\end{column}
\end{columns}
\end{block}

\end{frame}


\begin{frame}[fragile]
\frametitle{Algebraic datatypes and higher-order functions --- ADTs}

\begin{itemize}
\item Extremely easy to define, create, and use
\item Has a number of shorthands that make ADTs even more enjoyable to work with
\begin{itemize}
\medskip
\item Named enum fields
\begin{minted}{rust}
enum Sum<A, B>{
    Left{left_arg: A},
    Right{right_arg: B}
}
\end{minted}
\medskip
\item Named field punning
\begin{minted}{rust}
let first = ..; let second = ..;
let z = Pair{first, second};
\end{minted}
\medskip
\item Disjunctive patterns / "or-patterns"
\begin{minted}{rust}
match x {
    0 => ..,
    1 | 2 | 3 => ..,
    4 => ..,
    _ => ..
}
\end{minted}
\end{itemize}
\end{itemize}

\end{frame}


\begin{frame}[fragile]
\frametitle{Algebraic datatypes and higher-order functions --- HOFs}

Rust has anonymous functions, known as \emph{closures}.

\begin{columns}

\begin{column}[T]{0.48\textwidth}
\begin{block}{Creation}
\small
\begin{minted}{rust}
let f = |x| x + 1;
\end{minted}
\end{block}
\end{column}

\begin{column}[T]{0.48\textwidth}
\begin{block}{Use}
\small
\begin{minted}{rust}
let y = f(x);
\end{minted}
\end{block}
\end{column}

\end{columns}

\bigskip

Higher-order functions are fairly common in the standard library.

\begin{block}{Example: \href{https://doc.rust-lang.org/std/iter/struct.Map.html}{\texttt{map}}}
\begin{minted}{rust}
let xs: Vec<u32> = vec![1, 2, 3, 4];
let ys: Vec<u32> = xs.iter().map(|x| x + 1).collect();
\end{minted}
\end{block}

\end{frame}


\begin{frame}[fragile]
\frametitle{Algebraic datatypes and higher-order functions --- HOFs}

Working with closures / HOFs in Rust is more complex than in other languages.

\begin{block}{Haskell}
\begin{minted}{haskell}
map :: (a -> b) -> Maybe a -> Maybe b
maybe :: b -> (a -> b) -> Maybe a -> b
\end{minted}
\end{block}

\begin{block}{Rust}
\small
\begin{minted}{rust}
fn map<A, B, F: FnOnce(A) -> B>(
    f: F,
    value: Option<A>
) -> Option<B>

fn map_or<A, B, F: FnOnce(A) -> B>(
    default: B,
    f: F,
    value: Option<A>
) -> B
\end{minted}
\end{block}

\end{frame}


\begin{frame}[c,fragile]
\frametitle{Algebraic datatypes and higher-order functions --- HOFs}

Most languages have a single function type.

\bigskip

Rust has 3 kinds of closure:

\begin{itemize}
\item \href{https://doc.rust-lang.org/std/ops/trait.FnOnce.html}{\texttt{FnOnce}}
\item \href{https://doc.rust-lang.org/std/ops/trait.Fn.html}{\texttt{Fn}}
\item \href{https://doc.rust-lang.org/std/ops/trait.FnMut.html}{\texttt{FnMut}}
\end{itemize}

\note{
Most languages have a single function type. Rust has more. There are 3 kinds of
closure: \texttt{FnOnce}, \texttt{Fn}, and \texttt{FnMut}. I won't explain what they mean
right now.
Let me just say: they are all justified in the context of Rust's goals of safety and
performance. The price of achieving these goals for closures in Rust is simplicity:
when you write a higher-order function, you need to figure out which kind of
higher-order function is most appropriate.
}
\end{frame}


\begin{frame}[fragile]
\frametitle{Algebraic datatypes and higher-order functions}

Higher-order functions (in some form or another) seem like an essential component
of "functional programming".

\bigskip

Algebraic datatypes and higher-order functions are interdefinable:

\begin{itemize}
\item ADTs via HOFs --- \href{https://en.wikipedia.org/wiki/Church_encoding}{Church encoding}
\item HOFs via ADTs --- \href{https://en.wikipedia.org/wiki/Defunctionalization}{Defunctionalization}
\end{itemize}

\note{
It seems reasonable to me to claim that higher-order functions are an essential part
of functional programming.

\medskip

I've put algebraic datatypes and higher-order functions in the same section
because they're actually interdefinable. If you have one, then you have the other
(assuming you have first-order function definitions).

\medskip

Using higher-order functions to define algebraic datatypes is called Church encoding.

\medskip

Going the other way, algebraic datatypes (together with first-order functions) to
can be used to define higher-order functions through a method
called Defunctionalisation, which was the subject of Jack's recent talk, "Everything
looks like a function".
}
\end{frame}


\begin{frame}[fragile]
\frametitle{Immutability}

Variable bindings in Rust are immutable by default:

\begin{block}{Immutable by default}
\small
\begin{minted}{rust}
let x = 1;
x = x + 1; // error: cannot assign twice to immutable variable `x`
\end{minted}
\end{block}

\bigskip

Rust allows mutability, which comes in 3 flavours:

\begin{itemize}
\item Local mutability --- mild
\item Mutable references --- medium
\item Interior (hidden) mutability --- spicy
\end{itemize}
\end{frame}


\begin{frame}[c,fragile]
\frametitle{\st{Im}Mutability --- Local}

\begin{block}{Local mutability}
\begin{minted}{rust}
fn two() -> u32 {
    let mut x: u32 = 0;
    x += 1;
    x += 1;
    x
}
\end{minted}
\end{block}

\note{
This is the most benign form of mutability because it is compositional. When I call a
function, that function's use of local mutability is not a concern to me.
}
\end{frame}


\begin{frame}[c,fragile]
\frametitle{\st{Im}Mutability --- References}

\begin{block}{Definition}
\small
\begin{minted}{rust}
fn increment_immutable(x: &u32) {
    *x = x + 1; // error: cannot assign to `*x`, which is
                // behind a `&` reference
}

fn increment_mutable(x: &mut u32) {
    *x = x + 1; // ok
}
\end{minted}
\end{block}

\note{
I've lost some compositionality: if I'm not careful with my design then I could
create a history-sensitive function. A function that only works properly when I
call it in the right place at the right time.

\medskip

The upside is that because the mutability is in the type signature I'm reminded of
this possibility, so I can more consciously opting in to (potential) mutation when
I call such functions.
}
\end{frame}


\begin{frame}[c,fragile]
\frametitle{\st{Im}Mutability --- References}

\begin{block}{Use}
\small
\begin{minted}{rust}
fn increment_mutable(x: &mut u32) {
    *x = x + 1;
}

let x = 1;
increment_mutable(&x) // error: types differ in mutability

let x = 1;
increment_mutable(&mut x) // error: cannot borrow `x` as mutable,
                          // as it is not declared as mutable
let mut x = 1;
increment_mutable(&mut x) // ok
\end{minted}
\end{block}

\note{
You can only take a mutable reference to a mutable variable. I find adding the
\texttt{mut} keyword to a binding to be a helpful "confirmation" step for mindful
mutability.
}
\end{frame}


\begin{frame}[fragile]
\frametitle{\st{Im}Mutability --- Interior (hidden)}

Some types allow mutation through an "immutable" reference:

\begin{itemize}
\item \href{https://doc.rust-lang.org/std/cell/struct.Cell.html}{\texttt{Cell}}
\item \href{https://doc.rust-lang.org/std/cell/struct.RefCell.html}{\texttt{RefCell}}
\item \href{https://doc.rust-lang.org/std/sync/struct.Mutex.html}{\texttt{Mutex}}
\item \href{https://doc.rust-lang.org/std/sync/struct.RwLock.html}{\texttt{RwLock}}
\item \href{https://doc.rust-lang.org/std/sync/atomic/}{atomics}
\end{itemize}

\bigskip

\begin{tabular}{l l l l}
Type & Better Name & Multiplicity & Writable (\texttt{*x = ..}) \\
\hline
\texttt{\&T} & Shared reference & Many & No \\
\texttt{\&mut T} & Exclusive reference & One & Yes
\end{tabular}

\note{    
Some types allow mutation through an "immutable" reference. That's a hint that the
term "immutable reference" is a bit of a misnomer. Instead of talking about "immutable"
and "mutable" references, sometimes it's better to use the terms "shared" and "exclusive",
respectively.

\medskip

Rust prevents data races by only allowing writes via an exclusive reference.
}
\end{frame}


\begin{frame}[fragile]
\frametitle{\st{Im}Mutability --- Interior (hidden)}

\begin{block}{Example}
\begin{minted}{rust}
fn increment_interior(x: &Cell<u64>) -> u64 {
    let result = x.get() + x.get();
    x.set(x.get(x) + 1);
    result
}
\end{minted}
\end{block}

Mutability is no longer in the type signature.

\begin{block}{Interior mutability}
\begin{minted}{rust}
struct S{
    x: Cell<u64>
}

// Does `f_S` mutate its argument?
fn f_S(s: &S) -> { .. }
\end{minted}
\end{block}

\note{
When is interior mutability useful?

\bigskip

It's necessary for thread-shared mutable state, and that's the only time I've used it.
I've never used `Cell` or `RefCell` and I haven't thought about when they'd be justified.
}
\end{frame}


\begin{frame}[c,fragile]
\frametitle{Static types --- common features}

\begin{block}{Algebraic datatypes --- structs and enums}
\begin{minted}{rust}
struct TypeName{
  field_1: FieldType1,
  field_2: FieldType2,
  ...
}

enum TypeName{
  Constructor1(FieldType11, FieldType12, ...),
  Constructor2,
  ...,
}
\end{minted}
\end{block}

\end{frame}


\begin{frame}[c,fragile]
\frametitle{Static types --- common features}

\begin{block}{Parametric polymorphism --- generics}
\begin{minted}{rust}
struct TypeName<Param1, Param2>{ ... }
enum TypeName<Param1, Param2>{ ... }

fn id<A>(x: A) -> A { x }
\end{minted}
\end{block}

\end{frame}


\begin{frame}[c,fragile]
\frametitle{Static types --- common features}

\begin{block}{Ad-hoc polymorphism --- traits}
\begin{minted}{rust}
trait Monoid {
  fn empty() -> Self;
  fn combine(self, other: Self) -> Self
}

fn fold<M: Monoid>(xs: Vec<M>) -> M { .. }
\end{minted}
\end{block}

\end{frame}


\begin{frame}[c,fragile]
\frametitle{Static types --- uncommon features}

\begin{itemize}
\item Reference lifetime analysis (borrow checking)
\item Uniqueness types
\end{itemize}

\end{frame}

\begin{frame}[fragile]
\frametitle{Static types --- reference lifetime analysis}

Every reference points to valid memory.

\begin{columns}

\begin{column}[T]{0.48\textwidth}
\begin{block}{Rust}
\tiny
\begin{minted}{rust}
struct IntAndBool{
  first: i32,
  second: bool
}

fn first<'a>(input: &'a IntAndBool) -> &'a i32 {
  &input.first
}

fn bad<'a>() -> &'a i32 {
    let x = IntAndBool{first: 99, second: true};
    let y = first(&x);
    y
//  ^ error: returns a value referencing data
//    owned by the current function
}
\end{minted}
\end{block}
\end{column}

\begin{column}[T]{0.48\textwidth}
\begin{block}{C}
\tiny
\begin{minted}{c}
typedef struct {
  long first;
  bool second;
} IntAndBool;

long* first(IntAndBool* input) {
  return &input->first;
}

long* bad() {
  IntAndBool x = { .first = 99, .second = true };
  long* y = first(&x);
  return y; // ok...? 
}
\end{minted}
\end{block}
\end{column}

\end{columns}

\note{
Reference lifetime analysis, generally known as "borrow checking", determines at compile
time whether a program references data in a safe way.

\medskip

For example, you're allowed to get a refrence to a struct's field, pass it to and return
it from functions, store it in another datatype, etc. and the compiler will prevent you
from doing anything that could cause the reference to outlive the struct. In other words,
no dangling references / use-after-frees.
}
\end{frame}


\begin{frame}[fragile]
\frametitle{Static types --- uniqueness types}

\texttt{\&T} --- shared
  
\begin{columns}

\begin{column}[T]{0.48\textwidth}
\begin{block}{Concurrent use}
\small
\begin{minted}{rust}
fn f(x: &T, y: &T) { .. }

..

let x: T = ..;
f(&x, &x); // ok
\end{minted}
\end{block}
\end{column}

\begin{column}[T]{0.48\textwidth}
\begin{block}{Sequential use}
\small
\begin{minted}{rust}
fn g(x: &T) { .. }
fn h(x: &T) { .. }

..

let x: T = ..;
f(&x);
g(&x); // ok
\end{minted}
\end{block}
\end{column}

\end{columns}

\end{frame}


\begin{frame}[fragile]
\frametitle{Static types --- uniqueness types}

\texttt{\&mut T} --- exclusive
  
\begin{columns}

\begin{column}[T]{0.48\textwidth}
\begin{block}{Concurrent use}
\small
\begin{minted}{rust}
fn f(x: &mut T, y: &mut T) { .. }

..

let mut x: T = ..;
// error: cannot borrow `x` as
// mutable more than once at
// a time
f(&mut x, &mut x);
//        ^^^^^^ second mutable
//        borrow occurs here 
\end{minted}
\end{block}
\end{column}

\begin{column}[T]{0.48\textwidth}
\begin{block}{Sequential use}
\small
\begin{minted}{rust}
fn g(x: &mut T) { .. }
fn h(x: &mut T) { .. }

..

let mut x: T = ..;
f(&mut x);
g(&mut x); // ok
\end{minted}
\end{block}
\end{column}

\end{columns}

\end{frame}


\begin{frame}[fragile]
\frametitle{Static types --- uniqueness types}

\texttt{T} --- unique
  
\begin{columns}

\begin{column}[T]{0.48\textwidth}
\begin{block}{Concurrent use}
\small
\begin{minted}{rust}
fn f(x: T, y: T) { .. }

..

let x: T = ..;
// error: use of moved value `x`
f(x, x);
//   ^ value used here after move
\end{minted}
\end{block}
\end{column}

\begin{column}[T]{0.48\textwidth}
\begin{block}{Sequential use}
\small
\begin{minted}{rust}
fn g(x: T) { .. }
fn h(x: T) { .. }

..

let x: T = ..;
f(x);
// error: use of moved value `x`
g(x);
//^ value used here after move
\end{minted}
\end{block}
\end{column}

\end{columns}

\end{frame}


\begin{frame}[c,fragile]
\frametitle{Static types and functional programming}

\begin{itemize}
\item \href{https://en.wikipedia.org/wiki/Curry%E2%80%93Howard_correspondence}{Curry-Howard correspondence}: types are propositions and (lambda calculus) programs are proofs
\item Only holds in general when a type system is decidable
\item Rust's type system is (technically) undecidable
\begin{itemize}
\item \url{https://sdleffler.github.io/RustTypeSystemTuringComplete/}
\item \url{https://github.com/Ashymad/fortraith}
\end{itemize}
\end{itemize}

\note{
There's a deep theoretical reason that types are associated with functional programming.
The lambda calculus is a turing-complete model of computation which consists *only* of
functions. The definitive functional programming language. There are ways of defining
type systems for the lambda calculus that are equivalent to mathematical logics. In this
setting, a type is equivalent to a proposition, and a well-typed lambda calculus program
is a proof of that proposition. This is known as the Curry-Howard correspondence and I
think Donovan will be covering it in more detail next month. 

\medskip

Given this, some people might suggest that a functional programming language is one
that corresponds to a logic via Curry-Howard.

\medskip

The validity of "types as propositions and proofs as programs" for a programming
language requires type checking of that language to be decidable.

\medskip

Rust's type checker is known to be undecidable. It's possible to write Rust programs
that would take an infinite amount of time to type check. I think it's nearly impossible
in a typical software engineering context.
}
\end{frame}


\begin{frame}[fragile]
\frametitle{Equational reasoning}

Reasoning about / manipulating programs as if they are a set of equations.

\begin{columns}

\begin{column}[T]{0.48\textwidth}
\begin{block}{Factor out}
\begin{minted}{rust}
(y + 1) * (y + 1)
\end{minted}

\(\leadsto\)

\begin{minted}{rust}
let x = y + 1;
x * x
\end{minted}
\end{block}
\end{column}

\begin{column}[T]{0.48\textwidth}
\begin{block}{Inline}
\begin{minted}{rust}
let x = y + 1;
x * x
\end{minted}

\(\leadsto\)

\begin{minted}{rust}
(y + 1) * (y + 1)
\end{minted}
\end{block}
\end{column}

\end{columns}

\begin{columns}

\begin{column}[T]{0.24\textwidth}
\end{column}

\begin{column}[T]{0.48\textwidth}
\begin{block}{Simplify}
\begin{minted}{rust}
let y = x * 1 + 2
\end{minted}

\(\leadsto\)

\begin{minted}{rust}
let y = x + 2
\end{minted}
\end{block}
\end{column}

\begin{column}[T]{0.24\textwidth}
\end{column}

\end{columns}

\end{frame}


\begin{frame}[fragile]
\frametitle{Equational reasoning in Rust}
Not always possible in Rust.

\begin{columns}

\begin{column}[T]{0.48\textwidth}
\begin{block}{Example}
\begin{minted}{rust}
let mut x: u32 = 2;

let mut f = || -> u32 {
    x += 1;
    x
};

let y: u32 = f();
y + y
\end{minted}
\end{block}
\end{column}

\begin{column}[T]{0.48\textwidth}
\begin{block}{Execution}
\begin{minted}{rust}
let y = f();
y + y
\end{minted}

\(\leadsto\)

\begin{minted}{rust}
let y = 3;
y + y
\end{minted}

\(\leadsto\)

\begin{minted}{rust}
3 + 3
\end{minted}

\(\leadsto\)

\begin{minted}{rust}
6
\end{minted}
\end{block}
\end{column}

\end{columns}

\end{frame}


\begin{frame}[fragile]
\frametitle{Equational reasoning in Rust}
Not always possible in Rust.

\begin{columns}

\begin{column}[T]{0.48\textwidth}
\begin{block}{Example (inlined)}
\begin{minted}{rust}
let mut x: u32 = 2;

let mut f = || -> u32 {
    x += 1;
    x
};

// let y: u32 = f();
f() + f()
\end{minted}
\end{block}
\end{column}

\begin{column}[T]{0.48\textwidth}
\begin{block}{Execution (inlined)}
\begin{minted}{rust}
f() + f()
\end{minted}

\(\leadsto\)

\begin{minted}{rust}
3 + f()
\end{minted}

\(\leadsto\)

\begin{minted}{rust}
3 + 4
\end{minted}

\(\leadsto\)

\begin{minted}{rust}
7
\end{minted}
\end{block}
\end{column}

\end{columns}

\end{frame}


\begin{frame}[fragile]
\frametitle{Equational reasoning in Rust}

Equational reasoning fails when expressions depend on an implicit context.

\begin{block}{Example 1 --- read}
\begin{minted}{rust}
fn f(x: &mut u32) -> u32 { *x + 1 }

let x: &mut u32 = ..;
f(x)
\end{minted}
\end{block}

\begin{block}{Example 2 --- read-write}
\begin{minted}{rust}
fn f(x: &mut u32) -> u32 { *x += 1; 0 }

let x: &mut u32 = ..;
f(x)
\end{minted}
\end{block}

\note{
An expression that depends on an implicit context can evaluate to different results at
different points in the program.

\medskip

In example 1, the expression \texttt{f(x)}, applying the same function to the same reference,
could return different values at different points in the program. It depends on an
implicit mutable store.

\medskip

An expression that \emph{changes} an implicit context is not safe to inline, even if it
returns the same result every time --- see example 2. Inlining such expressions can change the meaning
of the program, because the program's meaning depends on how many times the expression
is evaluated.
}
\end{frame}


\begin{frame}
\frametitle{Equational reasoning in Rust}
Widespread equational reasoning is not a design goal for Rust.

\medskip

Many standard library functions are non-pure / have side-effects, e.g.:

\begin{itemize}
\item \href{https://doc.rust-lang.org/std/io/index.html}{Standard input / output}
\item \href{https://doc.rust-lang.org/std/env/index.html}{Environment variables}
\item \href{https://doc.rust-lang.org/stable/std/fs/index.html}{File system access}
\item \href{https://doc.rust-lang.org/std/net/index.html}{Networking}
\end{itemize}

\begin{block}{Terminology}
\begin{itemize}
\item An expression whose evaluation does not depend on an implicit context is described
as \emph{pure}.

\item A function that doesn't depend on an implicit context is a \emph{pure function}.

\item Changes to an implicit context are called \emph{side-effects}.
\end{itemize}
\end{block}

\note{
Mutable memory isn't the only implicit context in Rust programs. There is also the
program input / output, environment variables, file system, networking, among others.

\medskip

In Rust, the level of equational reasoning in a codebase depends on the programmers' judgement.
When writing new code, the author decides whether or not equational reasoning is relevant.
When refactoring, the programmer has to understand whether they are working with pure
or non-pure functions.
}
\end{frame}


\begin{frame}[fragile]
\frametitle{Equational reasoning and mutability}

Mutability doesn't always break equational reasoning.

\begin{block}{A pure function that uses mutability}
\begin{minted}{rust}
fn add(x: u32, y: u32) -> u32 {
    let mut counter = y;
    let mut result = x;
    
    while counter > 0 {
        counter -= 1;
        result += 1;
    }
    
    result
}
\end{minted}
\end{block}

\note{
I've showed a bunch of examples where mutability, in particular mutable references, breaks
equational reasoning. But mutability isn't \emph{intrinsically} opposed to equational reasoning.

Here is an example of a pure function that uses mutability internally. Equational reasoning
still holds.
}
\end{frame}


\begin{frame}[fragile]
\frametitle{Equational reasoning and mutability}
Rust's \emph{uniqueness types} offer another way to use mutation while
preserving equational reasoning.

\begin{block}{Example}
\begin{minted}{rust}
fn push<T>(mut xs: Vec<T>, x: T) -> Vec<T> {
    xs.push(x);
    xs
}

fn length<T>(xs: &Vec<T>) -> usize {
    xs.len()
}

let xs = vec![1, 2, 3]
let three = length(&xs);
let new_xs = push(xs, 4);
three + three
\end{minted}
\end{block}

\end{frame}


\begin{frame}[fragile]
\frametitle{Equational reasoning and mutability}
Rust's \emph{uniqueness types} offer another way to use mutation while
preserving equational reasoning.

\begin{block}{Evaluation}

\begin{minted}{rust}
let xs = vec![1, 2, 3]
let three = length(&xs);
let new_xs = push(xs, 4);
three + three
\end{minted}

\(\leadsto\)

\begin{minted}{rust}
let xs = /* pointer `p` to [1, 2, 3] */;
let three = length(&xs);
let new_xs = push(xs, 4);
three + three
\end{minted}

\(\leadsto\)

...

\end{block}

\end{frame}


\begin{frame}[fragile]
\frametitle{Equational reasoning and mutability}
Rust's \emph{uniqueness types} offer another way to use mutation while
preserving equational reasoning.

\begin{block}{Evaluation}

...

\(\leadsto\)

\begin{minted}{rust}
let three = 3; /* length(&xs) */
let new_xs = push(xs, 4);
three + three
\end{minted}

\(\leadsto\)

\begin{minted}{rust}
let new_xs = /* pointer `p` to [1, 2, 3, 4] */;
three + three
\end{minted}

\(\leadsto\)

\begin{minted}{rust}
3 + 3
\end{minted}

\(\leadsto\)

\begin{minted}{rust}
6
\end{minted}

\end{block}

\end{frame}


\begin{frame}[fragile]
\frametitle{Equational reasoning and mutability}
Rust's \emph{uniqueness types} offer another way to use mutation while
preserving equational reasoning.

\medskip

A unique variable is only in scope while it's guaranteed to remain unchanged.

\begin{block}{Example}
\begin{minted}{rust}
fn push<T>(mut xs: Vec<T>, x: T) -> Vec<T> { .. }

fn length<T>(xs: &Vec<T>) -> usize { .. }

let xs = vec![1, 2, 3]
// let three = length(&xs);
let new_xs = push(xs, 4);
// error: borrow of moved value `xs`
length(&xs) + length(&xs)
//     ^^^ value borrowed here after move
\end{minted}
\end{block}

\end{frame}


\begin{frame}[fragile]
\frametitle{Equational reasoning and mutability}
Rust's \emph{uniqueness types} offer another way to use mutation while
preserving equational reasoning.

\medskip

A unique variable is only in scope while it's guaranteed to remain unchanged.

\begin{block}{Example}
\begin{minted}{rust}
fn push<T>(mut xs: Vec<T>, x: T) -> Vec<T> { .. }

fn length<T>(xs: &Vec<T>) -> usize { .. }

let xs = vec![1, 2, 3]
// `xs` is immutable...
let new_xs = push(xs, 4); // ...until we actually change it
.. // after which we can no longer refer to `xs`
\end{minted}
\end{block}

\end{frame}


\begin{frame}[c]
\frametitle{Summary}

How Rust relates to:

\begin{itemize}
\item Algebraic datatypes and higher-order functions
\item Immutability
\item Static types
\item Equational reasoning
\end{itemize}

\end{frame}


\begin{frame}
\frametitle{Functional programming?}

\begin{itemize}
\item Focused on how we write and reason about programs
\begin{itemize}
\item Independent of specific program behaviours
\end{itemize}
\item Needs to be counterbalanced by other topics / concerns
\end{itemize}

\note<+>{
What is functional programming? I'm still going to refrain from a definition.

\bigskip

But I have a final observation to share. These four points I've discussed include
specific tools we use to write programs (language features), styles of code (preference
for immutability or mutability), and ways of reasoning about code. 

\bigskip

I think this isn't a coincidence: it seems to me that functional programming is largely
concerned with programs themselves, and the process of programming, as objects of study.
It is less concerned with the specific behaviours of any particular program.

\bigskip

Another way to put it is that FP seems to focus on the relationship between programmers
and programs, rather than the relationship between programs and the world.

\bigskip

I think this is super important and is also inherently risky. It means that the subject
of FP, like other reflective / meta-level subjects, is especially prone to "navel-gazing".
}

\note<+>{
If functional programming is less concerned with specific program behaviours, then it's
a necessarily incomplete answer to the question of how to create good software. I don't
think FP is going to tell me about the appropriate memory / CPU usage bounds for a program
and how to achieve them. Or how to create intuitive user interfaces or render photorealistic
3D graphics or make sure a drone stays level when it flies.

\bigskip

I think that in the past my definition of "good software" was missing a few dimensions,
which lead to me overestimating the important of "functional programming" in "good software".
Now I know that FP isn't a panacea and it doesn't have to be. Whatever it is, it's now one
of many tools of thought for pushing my software along a few of many dimensions of "good".
}
\end{frame}


\end{document}